\documentclass{article}
\usepackage{graphicx} 
\usepackage{enumitem}
\usepackage{kpfonts}
\usepackage{amssymb}
\newcommand{\abs}[1]{\lvert#1\rvert}
\newcommand{\norm}[1]{\lVert#1\rVert}
\providecommand{\sbrak}[1]{\ensuremath{{}\left[#1\right]}}
\providecommand{\brak}[1]{\ensuremath{\left(#1\right)}}
\providecommand{\cbrak}[1]{\ensuremath{\left\{#1\right\}}}
\newcommand{\myvec}[1]{\ensuremath{\begin{pmatrix}#1\end{pmatrix}}}
\newcommand{\myaugvec}[2]{\ensuremath{\begin{amatrix}{#1}#2\end{amatrix}}}
\newcommand{\mydet}[1]{\ensuremath{\begin{vmatrix}#1\end{vmatrix}}}
\begin{document}
\section*{NCERT 12.10.5.14}
\begin{enumerate}
    \item If $\vec{a},\vec{b},\vec{c}$ are mutually perpendicular vectors of equal magnitudes,show that the vector $\vec{a}+\vec{b}+\vec{c}$ is equally inclined to $\vec{a},\vec{b} and \vec{c}$.\\
    
    \textbf{Construction Steps}\\
    
    Since $\vec{a},\vec{b} and \vec{c}$ are mutually Perpendicular vectors, we have $\vec{a} \cdot \vec{b}=\vec{b} \cdot \vec{c}=\vec{c} \cdot \vec{a}=0$. It is given that : 
    $\mydet{\vec{a}}=\mydet{\vec{b}}=\mydet{\vec{c}}$
    let vector $\vec{a}+\vec{b}+\vec{c}$ be inclined to $\vec{a},\vec{b} and \vec{c}$ at angles $\theta_1,\theta_2 and \theta_3$ respectively.\\
    
    Then,we have:\\
    \begin{align}
        \cos{\theta_1}= \frac{\brak{\vec{a}+\vec{b}+\vec{c}} \cdot \vec{a}}{\mydet{\vec{a}+\vec{b}+\vec{c}}\mydet{\vec{a}}}\\
        =\frac{\vec{a} \cdot \vec{a} + \vec{b} \cdot \vec{a} + \vec{c} \cdot \vec{a}}{\mydet{\vec{a}+\vec{b}+\vec{c}}\mydet{\vec{a}}}\\
        =\frac{\mydet{\vec{a}}^2}{\mydet{\vec{a}+\vec{b}+\vec{c}}\mydet{\vec{a}}} \\
        \implies \frac{\mydet{\vec{a}}}{\mydet{\vec{a}+\vec{b}+\vec{c}}}
        \\ \cos{\theta_2}= \frac{\brak{\vec{a}+\vec{b}+\vec{c}} \cdot \vec{b}}{\mydet{\vec{a}+\vec{b}+\vec{c}}\mydet{\vec{b}}}\\
        =\frac{\vec{a} \cdot \vec{b} + \vec{b} \cdot \vec{b} + \vec{c} \cdot \vec{b}}{\mydet{\vec{a}+\vec{b}+\vec{c}}\mydet{\vec{b}}}\\
        =\frac{\mydet{\vec{b}}^2}{\mydet{\vec{a}+\vec{b}+\vec{c}}\mydet{\vec{b}}} \\
        \implies \frac{\mydet{\vec{b}}}{\mydet{\vec{a}+\vec{b}+\vec{c}}}\\
        \cos{\theta_3}= \frac{\brak{\vec{a}+\vec{b}+\vec{c}} \cdot \vec{c}}{\mydet{\vec{a}+\vec{b}+\vec{c}}\mydet{\vec{c}}}\\
        =\frac{\vec{a} \cdot \vec{c} + \vec{b} \cdot \vec{c} + \vec{c} \cdot \vec{c}}{\mydet{\vec{a}+\vec{b}+\vec{c}}\mydet{\vec{c}}}\\
        =\frac{\mydet{\vec{c}}^2}{\mydet{\vec{a}+\vec{b}+\vec{c}}\mydet{\vec{c}}} \\
        \implies \frac{\mydet{\vec{c}}}{\mydet{\vec{a}+\vec{b}+\vec{c}}}\\
    \end{align}\\
    now , as \begin{align}
        \mydet{\vec{a}} = \mydet{\vec{b}} = \mydet{\vec{c}},\\\cos{\theta_1} = \cos{\theta_2} = \cos{\theta_3}\\
        \therefore \theta_1=\theta_2=\theta_3
    \end{align}
    Hence, the vector $\brak{\vec{a}+\vec{b}+\vec{c}}$ is equally inclined to $\vec{a},\vec{b} and \vec{c}$\\
    
    let u assume that vector $\brak{\vec{a}+\vec{b}+\vec{c}}$ as $\vec{p}$ \\

    
    $\vec{a},\vec{b},\vec{c}$ points are $\brak{2,0,0},\brak{0,2,0},\brak{0,0,2} $  and $\brak{\vec{a}+\vec{b}+\vec{c}}$ is $\brak{2,2,2}$ \\

    
    let $O$ be the origin and points are $\brak{0,0,0}$ 

    les us consider the $\angle POA$ 
    \begin{align}
	\cos{\brak{POA}} = \frac{(P-O)^\top(A-O)}{\norm{P-O}\norm{A-O}}
\end{align}
Finding angle POA  \\
\begin{align}
	P-O &=\myvec{2\\2\\2}
\end{align}
\begin{align}
	A-O &=\myvec{2\\0\\0}
\end{align}
also calculating the values of norms
\begin{align}
	\norm{P-O} &= \sqrt{12} = 2\sqrt{3} \\
	\norm{A-O} &= \sqrt{4} = 2\\
 \cos{\brak{POA}} = \frac{(P-O)^\top(A-O)}{\norm{P-O}\norm{A-O}}\\
 \implies \frac{\myvec{2&2&2}\myvec{2\\0\\0} }{4\sqrt{3}}\\
 \implies \frac{1}{\sqrt{3}}\\
\angle POA = \cos^{-1}\brak{{\frac{1}{\sqrt{3}}}}
\end{align}
Similarly,$\angle POB,\angle POC$ are\\


Finding angle POB  \\
\begin{align}
	P-O &=\myvec{2\\2\\2}
\end{align}
\begin{align}
	B-O &=\myvec{0\\2\\0}
\end{align}
also calculating the values of norms
\begin{align}
	\norm{P-O} &= \sqrt{12} = 2\sqrt{3} \\
	\norm{B-O} &= \sqrt{4} = 2\\
 \cos{\brak{POB}} = \frac{(-O)^\top(B-O)}{\norm{P-O}\norm{B-O}}\\
 \implies \frac{\myvec{2&2&2}\myvec{0\\2\\0} }{4\sqrt{3}}\\
 \implies \frac{1}{\sqrt{3}}\\
\angle POB = \cos^-{1}\brak{{\frac{1}{\sqrt{3}}}}
\end{align}


Finding angle POC  \\
\begin{align}
	P-O &=\myvec{2\\2\\2}
\end{align}
\begin{align}
	C-O &=\myvec{0\\0\\2}
\end{align}
also calculating the values of norms
\begin{align}
	\norm{P-O} &= \sqrt{12} = 2\sqrt{3} \\
	\norm{C-O} &= \sqrt{4} = 2\\
 \cos{\brak{POA}} = \frac{(P-O)^\top(C-O)}{\norm{P-O}\norm{C-O}}\\
 \implies \frac{\myvec{2&2&2}\myvec{0\\0\\2} }{4\sqrt{3}}\\
 \implies \frac{1}{\sqrt{3}}\\
\angle POC = \cos^-{1}\brak{{\frac{1}{\sqrt{3}}}}
\end{align}
$\therefore \angle POA = \angle POB = \angle POC$ \\
Hence here all the angles are equal so they are equally inclined to each other
\end{enumerate}
\end{document}
P
