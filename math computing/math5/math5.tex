\documentclass[11pt]{book}
\usepackage{gvv-book}
\usepackage{gvv}
\usepackage[sectionbib,authoryear]{natbib}
\setcounter{secnumdepth}{3}
\setcounter{tocdepth}{2}

\begin{document}
\section*{NCERT 12.10.5.14}
\begin{enumerate}
    \item If $ \vec{A},\vec{B},\vec{C} $ are mutually perpendicular vectors of equal magnitudes,show that the  $ \vec{A}+\vec{B}+\vec{C} $ is equally inclined to $ \vec{A},\vec{B}  \text{and}  \vec{C} $.\\
    \textbf{Solution:}
    
    Suppose we have the following vectors:
    \begin{align*}
        \mathbf{v}_1 = \myvec{3 \\-3 \\0}  \\
        \mathbf{v}_2 = \myvec{0\\ 3\\ 2}  \\
        \mathbf{v}_3 = \myvec{-5\\-2\\ -1}
    \end{align*}
        

\textbf{Step 1: Initialize}

Set $\mathbf{u}_1 = \mathbf{v}_1$:\\

 $\mathbf{u}_1 = \myvec{3\\ -3\\ 0} $
 

\textbf{Step 2: Orthogonalization}

For  $ \mathbf{v}_2$ :
 \begin{align}
     \mathbf{u}_2 = \mathbf{v}_2 - \frac{\langle \mathbf{v}_2, \mathbf{u}_1 \rangle}{\langle \mathbf{u}_1, \mathbf{u}_1 \rangle} \mathbf{u}_1 \\
     \mathbf{u}_2=\mathbf{v}_2- \brak{\mathbf{u}_1 ^\top \mathbf{v}_2} \mathbf{u}_1\\ 
     \mathbf{u}_2 = \mathbf{v}_2 - \brak{-\frac{3}{2}} \mathbf{u}_1 
     \implies \myvec{1.5\\1.5\\2 }
 \end{align}

For $\mathbf{v}_3 $:
\begin{align}
    \mathbf{u}_3 = \mathbf{v}_3 - \frac{\langle \mathbf{v}_3, \mathbf{u}_1 \rangle}{\langle \mathbf{u}_1, \mathbf{u}_1 \rangle} \mathbf{u}_1 - \frac{\langle \mathbf{v}_3, \mathbf{u}_2 \rangle}{\langle \mathbf{u}_2, \mathbf{u}_2 \rangle} \mathbf{u}_2 \\
    \mathbf{u}_3=\mathbf{v}_3- \brak{\mathbf{u}_2 ^\top \mathbf{v}_3} \mathbf{u}_2- \brak{\mathbf{u}_1 ^\top \mathbf{v}_3} \mathbf{u}_1\\ 
\mathbf{u}_3 = \mathbf{v}_3 - \brak{-2.121} \mathbf{u}_1 - \brak{-4.28} \cdot \mathbf{u}_2 \\
\implies\myvec{-1.302\\ -1.302\\ 1.93} 
\end{align}

\textbf{Step 3: Normalization}

Normalize each vector:
\begin{align}
\mathbf{u}_1 = \frac{\mathbf{u}_1}{\norm{\mathbf{u}_1}} \\
\mathbf{u}_2 = \frac{\mathbf{u}_2}{\norm{\mathbf{u}_2}} \\
\mathbf{u}_3 = \frac{\mathbf{u}_3}{\norm{\mathbf{u}_3}} 
\end{align}

The final orthonormal basis is:
\begin{align*}
\mathbf{u}_1 = \myvec{0.707\\-0.707\\0}\\
\mathbf{u}_2 = \myvec{0.514\\0.514\\0.685}\\
\mathbf{u}_3 = \myvec{-0.487\\-0.487\\-0.724}\\
\end{align*}
\textbf{Step 4: QR Decoposition}

we calculate Q by means of Gram–Schmidt process\\
$Q$ is an orthogonal matrix 
\begin{align*}
    Q=\myvec{ 0.707&0.514&-0.487\\-0.707&0.514&-0.487\\0&0.685&-0.724}
\end{align*}
To verify it as a orthonormal matrix we have to check this property i.e,  $Q^{\top}.Q =I$
\begin{align*}
    \implies Q^\top Q &= \myvec{1&0&0\\0&1&0\\0&0&1}
\end{align*}
\textbf{Step 5: Findings angles $\vec{A},\vec{B},\vec{C} \text{and} \vec{A}+\vec{B}+\vec{C} $}
\begin{align}
    \vec{A}=\myvec{3 \\-3 \\0}\\
    \vec{B}=\myvec{0\\ 3\\ 2}  \\
    \vec{C}=\myvec{-5\\-2\\ -1}\\
    \vec{A}+\vec{B}+\vec{C}=\myvec{-2\\-2\\ 1}
\end{align}
Normalize each vector:\\
   \begin{align}
    \norm{\vec{A}}=\sqrt{9}=3\\
    \norm{\vec{B}}=\sqrt{13}\\
     \norm{\vec{C}}=\sqrt{30}\\
     \norm{\vec{A}+\vec{B}+\vec{C}}=\sqrt{9}=3
   \end{align}
Finding angles:
\begin{align}
    \cos{\theta_1}&=\frac{\myvec{3 \\-3 \\0}\myvec{-2&-2&1}}{\sqrt{9}\sqrt{9}}\\
    \cos{\theta_1}&=\frac{1}{9}\\
    \theta_1=\cos^{-1}\brak{\frac{1}{9}}
    \implies 84\degree
\end{align}
\begin{align}
    \cos{\theta_2}&=\frac{\myvec{0 \\3 \\2}\myvec{-2&-2&1}}{\sqrt{13}\sqrt{9}}\\
    \cos{\theta_2}&=\frac{-4}{3\sqrt{13}}\\
    \theta_2&=\cos^{-1}\brak{\frac{-4}{3\sqrt{13}}}
    \implies 111\degree
\end{align}
\begin{align}
    \cos{\theta_3}&=\frac{\myvec{-5 \\-2 \\-1}\myvec{-2&-2&1}}{\sqrt{30}\sqrt{9}}\\
    \cos{\theta_3}&=\frac{13}{\sqrt{30}\sqrt{9}}\\
    \theta_3&=\cos^{-1}\brak{\frac{13}{\sqrt{30}\sqrt{9}}}
    \implies 38\degree
\end{align}
\begin{align*}
    \therefore  \theta_1\neq\theta_2\neq\theta_3
\end{align*}
Hence  we can say that $\vec{A}+\vec{B}+\vec{C} $ is not equally inclined to $\vec{A},\vec{B} \text{and} \vec{C}  $ 
\end{enumerate}
\end{document}
