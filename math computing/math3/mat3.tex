\documentclass[12pt,-letter paper]{article}
\usepackage{siunitx}
\usepackage{setspace}
\usepackage{gensymb}
\usepackage{xcolor}
\usepackage{caption}
%\usepackage{subcaption}
\doublespacing
\singlespacing
\usepackage[none]{hyphenat}
\usepackage{amssymb}
\usepackage{relsize}
\usepackage[cmex10]{amsmath}
\usepackage{mathtools}
\usepackage{amsmath}
\usepackage{commath}
\usepackage{amsthm}
\interdisplaylinepenalty=2500
%\savesymbol{iint}
\usepackage{txfonts}
%\restoresymbol{TXF}{iint}
\usepackage{wasysym}
\usepackage{amsthm}
\usepackage{mathrsfs}
\usepackage{txfonts}
\let\vec\mathbf{}
\usepackage{stfloats}
\usepackage{float}
\usepackage{cite}
\usepackage{cases}
\usepackage{subfig}
%\usepackage{xtab}
\usepackage{longtable}
\usepackage{multirow}
%\usepackage{algorithm}
\usepackage{amssymb}
%\usepackage{algpseudocode}
\usepackage{enumitem}
\usepackage{mathtools}
%\usepackage{eenrc}
%\usepackage[framemethod=tikz]{mdframed}
\usepackage{listings}
%\usepackage{listings}
\usepackage[latin1]{inputenc}
%%\usepackage{color}{   
%%\usepackage{lscape}
\usepackage{textcomp}
\usepackage{titling}
\usepackage{hyperref}
%\usepackage{fulbigskip}   
\usepackage{tikz}
\usepackage{graphicx}
\lstset{
  frame=single,
  breaklines=true
}
\let\vec\mathbf{}
\usepackage{enumitem}
\usepackage{graphicx}
\usepackage{siunitx}
\let\vec\mathbf{}
\usepackage{enumitem}
\usepackage{graphicx}
\usepackage{enumitem}
\usepackage{tfrupee}
\usepackage{amsmath}
\usepackage{amssymb}
\usepackage{mwe} % for blindtext and example-image-a in example
\usepackage{wrapfig}
\graphicspath{{figs/}}
\providecommand{\cbrak}[1]{\ensuremath{\left\{#1\right\}}}
\providecommand{\brak}[1]{\ensuremath{\left(#1\right)}}
\providecommand{\norm}[1]{\left\lVert#1\right\rVert}
\newcommand{\myvec}[1]{\ensuremath{\begin{pmatrix}#1\end{pmatrix}}}
\newcommand{\augvec}[3]{\ensuremath{\begin{amatrix}{#1|#2}#3\end{amatrix}}}
\newcommand{\mydet}[1]{\ensuremath{\begin{vmatrix}#1\end{vmatrix}}}
\begin{document}
\section*{NCERT 12.10.5.14}
\begin{enumerate}
    \item If $ \vec{A},\vec{B},\vec{C} $ are mutually perpendicular vectors of equal magnitudes,show that the vector $ \vec{A}+\vec{B}+\vec{C} $ is equally inclined to $ \vec{A},\vec{B} and \vec{C} $.\\
    
    \textbf{Construction Steps}\\
    
    Since $\vec{A},\vec{B} and \vec{C}$ are mutually Perpendicular vectors, we have $\vec{A} \cdot \vec{B}=\vec{B} \cdot \vec{C}=\vec{C} \cdot \vec{A}=0$.\\
    It is given that : 
    \begin{align*}
         \mydet{\vec{A}}=\mydet{\vec{B}}=\mydet{\vec{C}}
    \end{align*}
   
    let vector $\vec{A}+\vec{B}+\vec{C}$ be inclined to $\vec{A}, \vec{B} and \vec{C}$ at angles $\theta_1,\theta_2 and \theta_3$ respectively.\\
    
    Then,we have:\\

    \begin{align}
			\norm{\vec{B}-\vec{A}} \triangleq \sqrt{\brak{\vec{B}-\vec{A}}^{\top}{\vec{B}-\vec{A}}}
		\end{align}
    let u assume that vector $\brak{\vec{A}+\vec{B}+\vec{C}}$ as $\vec{P}$ \\

    
    $\vec{A},\vec{B},\vec{C}$ points are $\brak{2,0},\brak{2,0},\brak{2,0} $ in three different axis and $\brak{\vec{A}+\vec{B}+\vec{C}}$ is $\brak{2,2}$ \\

    
    let $\vec{O}$ be the origin and points are $\brak{0,0}$ 

    let us consider the $\angle \vec{POA}$ 
  
    \begin{align}
	\cos{\brak{\theta_1}} = \frac{(\vec{P}-\vec{O})^\top(\vec{A}-\vec{O})}{\norm{\vec{P}-\vec{O}}\norm{\vec{A}-\vec{O}}}
\end{align}
Finding angle POA  \\
\begin{align}
	\vec{P}-\vec{O} &=\myvec{2\\2}
\end{align}
\begin{align}
	\vec{A}-\vec{O} &=\myvec{2\\0}
\end{align}
\\we know that 
 \begin{align}
 \begin{split}
     \norm{\vec{P}-\vec{O}} \triangleq \sqrt{\brak{\vec{P}-\vec{O}}^{\top}{\vec{P}-\vec{O}}}\\
     \norm{\vec{A}-\vec{O}} \triangleq \sqrt{\brak{\vec{A}-\vec{O}}^{\top}{\vec{A}-\vec{O}}}
 \end{split}
\end{align}
\begin{align}
	\norm{\vec{P}-\vec{O}} &= \sqrt{8} = 2\sqrt{2} \\
	\norm{\vec{A}-\vec{O}} &= \sqrt{4} = 2\\
 \cos{\brak{\theta_1}} = \frac{(\vec{P}-\vec{O})^\top(\vec{A}-\vec{O})}{\norm{\vec{P}-\vec{O}}\norm{\vec{A}-\vec{O}}}\\
 \implies \frac{\myvec{2&2}\myvec{2\\0} }{4\sqrt{2}}\\
 \implies \frac{1}{\sqrt{2}}\\
	\angle POA = \cos^{-1}\brak{{\frac{1}{\sqrt{2}}}}
\end{align}


let us consider the $\angle \vec{POB}$ 
  
    \begin{align}
	\cos{\brak{\theta_2}} = \frac{(\vec{P}-\vec{O})^\top(\vec{B}-\vec{O})}{\norm{\vec{P}-\vec{O}}\norm{\vec{B}-\vec{O}}}
\end{align}
Finding angle $\vec{POB}$  \\
\begin{align}
	\vec{P}-\vec{O} &=\myvec{2\\2}
\end{align}
\begin{align}
	\vec{B}-\vec{O} &=\myvec{2\\0}
\end{align}
\\we know that 
 \begin{align}
 \begin{split}
     \norm{\vec{P}-\vec{O}} \triangleq \sqrt{\brak{\vec{P}-\vec{O}}^{\top}{\vec{P}-\vec{O}}}\\
     \norm{\vec{B}-\vec{O}} \triangleq \sqrt{\brak{\vec{B}-\vec{O}}^{\top}{\vec{B}-\vec{O}}}
 \end{split}
		\end{align}
\begin{align}
	\norm{\vec{P}-\vec{O}} &= \sqrt{8} = 2\sqrt{2} \\
	\norm{\vec{B}-\vec{O}} &= \sqrt{4} = 2\\
 \cos{\brak{\theta_2}} = \frac{(\vec{P}-\vec{O})^\top(\vec{B}-\vec{O})}{\norm{\vec{P}-\vec{O}}\norm{\vec{B}-\vec{O}}}\\
 \implies \frac{\myvec{2&2}\myvec{2\\0} }{4\sqrt{2}}\\
 \implies \frac{1}{\sqrt{2}}\\
\angle \vec{POB} = \cos^{-1}\brak{{\frac{1}{\sqrt{2}}}}
\end{align}


let us consider the $\angle \vec{POC}$ 
  
    \begin{align}
	\cos{\brak{\theta_3}} = \frac{(\vec{P}-\vec{O})^\top(\vec{C}-\vec{O})}{\norm{\vec{P}-\vec{O}}\norm{\vec{C}-\vec{O}}}
\end{align}
Finding angle $\vec{POC}$  \\
\begin{align}
	\vec{P}-\vec{O} &=\myvec{2\\2}
\end{align}
\begin{align}
	\vec{C}-\vec{O} &=\myvec{2\\0}
\end{align}
\\we know that 
 \begin{align}
 \begin{split}
     \norm{\vec{P}-\vec{O}} \triangleq \sqrt{\brak{\vec{P}-\vec{O}}^{\top}{\vec{P}-\vec{O}}}\\
     \norm{\vec{C}-\vec{O}} \triangleq \sqrt{\brak{\vec{C}-\vec{O}}^{\top}{\vec{C}-\vec{O}}}
 \end{split}
		\end{align}
\begin{align}
	\norm{\vec{P}-\vec{O}} &= \sqrt{8} = 2\sqrt{2} \\
	\norm{\vec{C}-\vec{O}} &= \sqrt{4} = 2\\
 \cos{\brak{\theta_3}} = \frac{(\vec{P}-\vec{O})^\top(\vec{C}-\vec{O})}{\norm{\vec{P}-\vec{O}}\norm{\vec{C}-\vec{O}}}\\
 \implies \frac{\myvec{2&2}\myvec{2\\0} }{4\sqrt{2}}\\
 \implies \frac{1}{\sqrt{2}}\\
\angle \vec{POC} = \cos^{-1}\brak{{\frac{1}{\sqrt{2}}}}
\end{align}

$\therefore \angle \vec{POA} = \angle \vec{POB} = \angle \vec{POC}$ \\
Hence here all the angles are equal so they are equally inclined to each other
\end{enumerate}
\end{document}

