\documentclass[11pt]{book}
\usepackage{gvv-book}
\usepackage{gvv}
\usepackage[sectionbib,authoryear]{natbib}
\setcounter{secnumdepth}{3}
\setcounter{tocdepth}{2}

\begin{document}
\section*{NCERT 12.10.5.14}
\begin{enumerate}
    \item If $ \vec{A},\vec{B},\vec{C} $ are mutually perpendicular vectors of equal magnitudes,show that the  $ \vec{A}+\vec{B}+\vec{C} $ is equally inclined to $ \vec{A},\vec{B}  \text{ and }  \vec{C} $.\\
    \textbf{Solution:}
    
    Suppose we have the following vectors:
    \begin{align*}
        \vec{v}_1 = \myvec{1 \\1 \\1}  \\
        \vec{v}_2 = \myvec{6\\ 4\\ 5}  \\
        \vec{v}_3 = \myvec{3 \\6\\ 9}
    \end{align*}
        

\textbf{Step 1: Initialize}

Set $\vec{u}_1 = \vec{v}_1$:\\

 $\vec{u}_1 = \myvec{1\\1\\1}$
 

\textbf{Step 2: Orthogonalization}

For  $ \vec{v}_2$ :
 \begin{align}
     \vec{u}_2 = \vec{v}_2 - \frac{\langle \vec{v}_2, \vec{u}_1 \rangle}{\langle \vec{u}_1, \vec{u}_1 \rangle} \vec{u}_1 \\
     \vec{u}_2=\vec{v}_2- \brak{\vec{u}_1 ^\top \vec{v}_2} \vec{u}_1\\ 
     \vec{u}_2\implies \myvec{1\\-1\\0}
 \end{align}

For $\vec{v}_3 $:
\begin{align}
    \vec{u}_3 = \vec{v}_3 - \frac{\langle \vec{v}_3, \vec{u}_1 \rangle}{\langle \vec{u}_1, \vec{u}_1 \rangle} \vec{u}_1 - \frac{\langle \vec{v}_3, \vec{u}_2 \rangle}{\langle \vec{u}_2, \vec{u}_2 \rangle} \vec{u}_2 \\
    \vec{u}_3=\vec{v}_3- \brak{\vec{u}_2 ^\top \vec{v}_3} \vec{u}_2- \brak{\vec{u}_1 ^\top \vec{v}_3} \vec{u}_1\\ 
\vec{u}_3\implies \myvec{-1\\-1\\2}
\end{align}

\textbf{Step 3: Normalization}

Normalize each vector:
\begin{align}
\vec{u}_1 = \frac{\vec{u}_1}{\norm{\vec{u}_1}} \\
\vec{u}_2 = \frac{\vec{u}_2}{\norm{\vec{u}_2}} \\
\vec{u}_3 = \frac{\vec{u}_3}{\norm{\vec{u}_3}} 
\end{align}

The final orthonormal basis is:
\begin{align*}
\vec{u}_1 = \myvec{\frac{1}{\sqrt{3}}\\ \frac{1}{\sqrt{3}}\\ \frac{1}{\sqrt{3}}} 
\implies\myvec{0.577\\0.577\\0.577}\\
\vec{u}_2 = \myvec{\frac{1}{\sqrt{2}}\\\frac{-1}{\sqrt{2}}\\0 }
\implies \myvec{0.707 \\ -0.707 \\ 0}\\
\vec{u}_3 = \myvec{\frac{-1}{\sqrt{6}}\\ \frac{-1}{\sqrt{6}}\\ \frac{2}{\sqrt{6}}} 
\implies \myvec{-0.408\\-0.408\\0.816}
\end{align*}
\textbf{Step 4: QR Decoposition}

we calculate Q by means of Gram–Schmidt process\\
$Q$ is an orthogonal matrix 
\begin{align*}
    Q=\myvec{ 0.577&0.707&-0.408\\0.577&-0.707&-0.408\\0.577&0&0.816}
\end{align*}
To verify it as a orthonormal matrix we have to check this property i.e,  $Q^{\top}.Q =I$
\begin{align*}
    \implies Q^\top Q &= \myvec{1&0&0\\0&1&0\\0&0&1}
\end{align*}
\textbf{Step 5: Findings angles between $\vec{u}_1,\vec{u}_2,\vec{u}_3  \text{ and } \vec{u}_1+\vec{u}_2+\vec{u}_3 $}
\begin{align}
    \vec{u}_1=\myvec{0.577\\0.577\\0.577}\\
    \vec{u}_2=\myvec{0.707 \\ -0.707 \\ 0} \\
    \vec{u}_3=\myvec{-0.408\\-0.408\\0.816}\\
    \vec{u}_1+\vec{u}_2+\vec{u}_3\implies\vec{y}=\myvec{0.876\\-0.538\\1.393}
\end{align}
Normalize each vector:\\
   \begin{align}
    \norm{\vec{u}_1}=0.9987\\
    \norm{\vec{u}_2}=0.9998\\
     \norm{\vec{u}_3}=0.9987\\
     \norm{\vec{y}}=2.9972
   \end{align}
Finding angles:
\begin{align}
    \cos{\theta_1}&=\frac{\myvec{0.577 \\0.577 \\0.577}\myvec{0.876&-0.538&1.393}}{\brak{0.9987}\brak{2.9972}}\\
    \cos{\theta_1}&=\frac{1}{3}\\
    \theta_1=\cos^{-1}\brak{\frac{1}{3}}
    \implies 70\degree
\end{align}
\begin{align}
   \cos{\theta_1}&=\frac{\myvec{0.707 \\-0.707 \\0}\myvec{0.876&-0.538&1.393}}{\brak{0.99967}\brak{2.9972}}\\
    \cos{\theta_1}&=\frac{1}{3}\\
    \theta_1=\cos^{-1}\brak{\frac{1}{3}}
    \implies 70\degree
\end{align}
\begin{align}
     \cos{\theta_1}&=\frac{\myvec{-0.408 \\-0.408\\0.816}\myvec{0.876&-0.538&1.393}}{\brak{0.9987}\brak{2.9972}}\\
    \cos{\theta_1}&=\frac{1}{3}\\
    \theta_1=\cos^{-1}\brak{\frac{1}{3}}
    \implies 70\degree
\end{align}
\begin{align*}
    \therefore  \theta_1=\theta_2=\theta_3
\end{align*}
Hence  we can say that $\vec{u}_1+\vec{u}_2+\vec{u}_3 $ is equally inclined to $\vec{u}_1,\vec{u}_2 \text{and} \vec{u}_3  $ 
\end{enumerate}
\end{document}
